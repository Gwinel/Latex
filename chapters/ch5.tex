\chapter{章标题}
\section{节标题}

\subsection{子节标题}
%插图图
% \begin{figure}[htbp]
% \centering
% \includegraphics[width=0.65\textwidth]{bloomfilter}
% \caption{布隆过滤器}\label{fig:bf}
% \end{figure}

%定义
\begin{definition}
	假阳性(false positives)也叫误判是指当前元素不在集合内,但由于哈希冲突的缘故存在其它元素被映射到部分相同bit位上,从而有一定的概率导致在判定该元素时认定其对应的所有位置都为1,从而判定其在集合内,造成一次误判。这个概率本文称为误判率,误判率用\begin{math}\epsilon\end{math}表示。
\end{definition}

%公式
% \begin{equation}
% (1 - {1/2})^k = 2^{-ln2\cdot\frac{m}{n}} \thickapprox 0.62{m/n}
% \end{equation}

%表
% \begin{table}[htbp]
%   \centering
%   \caption{相关符号及其含义}
%   \label{tab:ckf_para}
%   \begin{tabular}{cc}
%     \toprule
%       参数  & 含义  \\
%     \midrule
%       $\epsilon$  				&   误判率 \\
%         f 							  &   指纹信息的长度(单位:bit)\\
%       $\alpha$					  &   负载因子($0\leq\alpha\leq 1$) \\
%     	b                   &   每个哈希桶内包含的实体的数量 \\
%     	m 							    &	  哈希桶的数量 \\
%     	n                   &   元素的数量 \\
%       C 							    &   表达一个元素所需的平均位数(单位:bit)\\
%     \bottomrule
%   \end{tabular}
% \end{table}

\subsection{加锁与解锁}
接下来的内容介绍如何在论文中插入伪代码,仅供参考。
\begin{algorithm}[htbp]
\textbf{struct} \{\\
        ~~~~mcs\_lock\_t    lock;\\
        ~~~~\textbf{uint32\_t}    pad1[128/4 - sizeof(mcs\_lock\_t)];\\
\} spec\_mcs\_lock\_t;\\

\textbf{struct} \{\\
        ~~~~mcs\_lock\_t        mcs;\\
        ~~~~\textbf{uint8\_t} mode;\\
       ~~~~\textbf{char} padding[];\\
\} locklib\_mutex\_t;\\

\textbf{int} locklib\_mutex\_lock~(locklib\_mutex\_t ~*mutex, \textbf{uint8\_t} mode)\{\\
        ~~~~spec\_mcs\_lock\_t~ *lock ~= ~(spec\_mcs\_lock\_t ~*) ~mutex;\\
        ~~~~\textbf{uint32\_t} reason = 0;\\

speculative\_path:\\
        ~~~~XBEGIN(fallback\_path, reason);\\
        ~~~~\textbf{if}~ (lock->lock) ~XABORT(1); \\
        ~~~~~~~~\textbf{return} 0;\\

fallback\_path:\\
        ~~~~retries++;\\
        ~~~~\textbf{while}~ (lock->lock)\\
                ~~~~~~~~cpu\_relax();\\
        ~~~~\textbf{if} (retries < MAX\_RETRIES)\\
                ~~~~~~~~\textbf{goto} speculative\_path;\\

        \tcc{标准方式申请锁}\ 
        ~~~~my\_node.locked = true;\\
        ~~~~qnode\_t *prev = \_\_sync\_lock\_TAS(\&lock->lock, \&my\_node);\\
        ~~~~\textbf{if} ~(unlikely(prev != NULL)) \{\\
                ~~~~~~~~prev->next = \&my\_node;\\
                ~~~~~~~~\textbf{while}~ (my\_node.locked)\\
                        ~~~~~~~~~~~~cpu\_relax();\\
        ~~~~\}\\
        ~~~~mutex->mode = mode;\\
        ~~~~\textbf{return} 0; \\
\}\\
\caption{基于Intel RTM的MCS锁算法}
\label{algo:mcs_lock}
\end{algorithm}
